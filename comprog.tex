\documentclass[11pt,a4paper,titlepage]{article}
\usepackage[utf8]{inputenc}
\usepackage[T1]{fontenc}
\usepackage{amsmath, amsthm, amssymb, amsfonts}
\usepackage{graphicx}
\usepackage{geometry}
\usepackage{fancyhdr}
\usepackage{booktabs}
\usepackage{float}
\usepackage[vlined,ruled]{algorithm2e}
%\usepackage{hyperref}
\usepackage{color}
\usepackage{caption}
\usepackage{subcaption}
\usepackage{listings}

\usepackage[T1]{fontenc}
\usepackage[scaled]{beramono}

\definecolor{bluekeywords}{rgb}{0.13,0.13,1}
\definecolor{greencomments}{rgb}{0,0.5,0}
\definecolor{redstrings}{rgb}{0.9,0,0}

\usepackage{listings}
\lstset{language=C++,
numbers=left,
showspaces=false,
showtabs=false,
breaklines=true,
showstringspaces=false,
breakatwhitespace=true,
escapeinside={(*@}{@*)},
commentstyle=\color{greencomments},
keywordstyle=\color{bluekeywords}\bfseries,
stringstyle=\color{redstrings},
basicstyle=\ttfamily
}

% Header/Footer
\geometry{includeheadfoot, margin=2.5cm}
\pagestyle{fancy}
\chead{}
\rhead{\small \textsc{Competitive Programming}}
\cfoot{\thepage}
\setlength{\headheight}{15.2pt}
\renewcommand{\headrulewidth}{0.4pt}
\renewcommand{\footrulewidth}{0.4pt}


% New math operators
\DeclareMathOperator{\lcm}{lcm}


% Title/Author
\title{Competitive Programming}
\author{Bjarki Ágúst Guðmundsson \and Trausti Sæmundsson \and Ingólfur Eðvarðsson}


\begin{document}

	\maketitle
	\tableofcontents
	\newpage

	\section{Data Structures}

		\subsection{Union-Find}
			\lstinputlisting{code/data-structures/unionfind.cpp}

		\subsection{Segment Tree}
		\subsection{Fenwick Tree}
		\subsection{Interval Tree}

	\section{Graphs}

		\subsection{Breadth-First Search}

			An implementation of a breadth-first search that counts the number of edges on the shortest path from the starting vertex to the ending vertex in the specified unweighted graph (which is represented with an adjacency list). Note that it assumes that the two vertices are connected.
			\lstinputlisting{code/graph/bfs.cpp}

			Another implementation that doesn't assume the two vertices are connected. If there is no path from the starting vertex to the ending vertex, a \lstinline$-1$ is returned.
			\lstinputlisting{code/graph/bfs_visited.cpp}

		\subsection{Depth-First Search}
		\subsection{Single Source Shortest Path}
			\subsubsection{Dijkstra's algorithm}
				An implementation of Dijkstra's algorithm that returns the length of the shortest path from the starting vertex to the ending vertex.
				\lstinputlisting{code/graph/dijkstra.cpp}

				Another implementation that returns a map, where each key of the map is a vertex reachable from the starting vertex, and the value is a tuple of the length from the starting vertex to the current vertex and the vertex that preceeds the current vertex on the shortest path from the starting vertex to the current vertex.
				\lstinputlisting{code/graph/dijkstra_path.cpp}
			\subsubsection{Bellman-Ford algorithm}
		\subsection{All Pairs Shortest Path}
			\subsubsection{Floyd-Warshall algorithm}
		\subsection{Connected Components}
			\subsubsection{Modified Breadth-First Search}
		\subsection{Strongly Connected Components}
			\subsubsection{Kosaraju's algorithm}
			\subsubsection{Tarjan's algorithm}
		\subsection{Topological Sort}
			\subsubsection{Modified Breadth-First Search}
		\subsection{Articulation Points/Bridges}
			\subsubsection{Modified Depth-First Search}

	\section{Mathematics}

		\subsection{Binomial Coefficients}
			The binomial coefficient $\binom{n}{k} = \frac{n!}{k!(n-k)!}$ is the number of ways to choose $k$ items out of a total of $n$ items.

			\lstinputlisting{code/mathematics/nck_1.cpp}

			\lstinputlisting{code/mathematics/nck_2.cpp}

			\lstinputlisting{code/mathematics/nck_3.cpp}

		\subsection{Euclidean algorithm}
			The Euclidean algorithm computes the greatest common divisor of two integers $a$, $b$.
			\lstinputlisting{code/mathematics/gcd.cpp}

			The extended Euclidean algorithm computes the greatest common divisor $d$ of two integers $a$, $b$ and also finds two integers $x$, $y$ such that $a\times x + b\times y = d$.
			\lstinputlisting{code/mathematics/egcd.cpp}

		\subsection{Modular Multiplicative Inverse}
			\lstinputlisting{code/mathematics/mod_inv.cpp}

		\subsection{Modular Exponentiation}
			\lstinputlisting{code/mathematics/mod_pow.cpp}

		\subsection{Chinese Remainder Theorem}
			\lstinputlisting{code/mathematics/crt.cpp}

\end{document}