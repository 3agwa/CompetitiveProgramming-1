\documentclass[8pt,a4paper]{amsart}
\usepackage{amsmath, amsthm, amssymb, amsfonts}
\usepackage[T1]{fontenc}
\usepackage[utf8]{inputenc}
\usepackage{booktabs}
\usepackage{caption}
\usepackage{color}
\usepackage{fancyhdr}
\usepackage{float}
\usepackage{fullpage}
\usepackage{geometry}
\usepackage{graphicx}
% \usepackage{listings}
\usepackage{subcaption}
\usepackage[scaled]{beramono}
\usepackage[vlined,ruled]{algorithm2e}

% Listings
% \definecolor{bluekeywords}{rgb}{0.13,0.13,1}
% \definecolor{greencomments}{rgb}{0,0.5,0}
% \definecolor{redstrings}{rgb}{0.9,0,0}
% \lstset{language=C++,
%     numbers=left,
%     showspaces=false,
%     showtabs=false,
%     breaklines=true,
%     showstringspaces=false,
%     breakatwhitespace=true,
%     escapeinside={(*@}{@*)},
%     commentstyle=\color{greencomments},
%     keywordstyle=\color{bluekeywords}\bfseries,
%     stringstyle=\color{redstrings},
%     basicstyle=\ttfamily
% }

% Minted
\usepackage{minted}
\newcommand{\code}[1]{\inputminted[linenos]{cpp}{_code/#1}}

% Header/Footer
\geometry{includeheadfoot, margin=2.5cm}
\pagestyle{fancy}
\chead{}
\rhead{}
\cfoot{\thepage}
\setlength{\headheight}{15.2pt}
\renewcommand{\headrulewidth}{0.4pt}
\renewcommand{\footrulewidth}{0.4pt}


% Math and bit operators
\DeclareMathOperator{\lcm}{lcm}
\newcommand*\BitAnd{\mathrel{\&}}
\newcommand*\BitOr{\mathrel{|}}
\newcommand*\ShiftLeft{\ll}
\newcommand*\ShiftRight{\gg}
\newcommand*\BitNeg{\ensuremath{\mathord{\sim}}}

% Title/Author
\title{Algorithms and Data Structures for Competitive Programming}

\begin{document}

\maketitle
\tableofcontents
\newpage

\section{Code Templates}
    \subsection{Basic Configuration}
        Vim and (Caps Lock = Escape) configuration.
        \begin{minted}{bash}
xset r rate 150 100
xmodmap -e "remove Lock = Caps_Lock" -e "keysym Caps_Lock = Escape" -e "add Lock = Caps_Lock"
echo "set nocp et sw=4 ts=4 sts=4 si cindent ru noeb showcmd showmode | syn on | colorscheme slate" > ~/.vimrc
        \end{minted}

    \subsection{C++ Header}
        A C++ header.
        \code{header.cpp}

    \subsection{Java Template}
        A Java template.
        \code{template.java}


\section{Data Structures}

    \subsection{Union-Find}
        An implementation of the Union-Find disjoint sets data structure.
        \code{data-structures/union_find.cpp}

    \subsection{Segment Tree}
        An implementation of a Segment Tree.
        \code{data-structures/segment_tree.cpp}

    \subsection{Fenwick Tree}
        A Fenwick Tree is a data structure that represents an array of $n$
        numbers. It supports adjusting the $i$-th element in $O(\log n)$ time,
        and computing the sum of numbers in the range $i..j$ in $O(\log n)$
        time. It only needs $O(n)$ space.
        \code{data-structures/fenwick_tree.cpp}

    \subsection{Matrix}
        A Matrix class.
        \code{data-structures/matrix.cpp}

    \subsection{Trie}
        A Trie class.
        \code{data-structures/trie.cpp}

    \subsection{AVL Tree}
        A fast, easily augmentable, balanced binary search tree.
        \code{data-structures/avl_tree.cpp}

        Also a very simple wrapper over the AVL tree that implements a map
        interface.
        \code{data-structures/avl_map.cpp}

    \subsection{Heap}
        An implementation of a binary heap.
        \code{data-structures/heap.cpp}

    \subsection{Skiplist}
        An implementation of a skiplist.
        \code{data-structures/skiplist.cpp}

\section{Graphs}
    \subsection{Breadth-First Search}
        An implementation of a breadth-first search that counts the number of
        edges on the shortest path from the starting vertex to the ending
        vertex in the specified unweighted graph (which is represented with an
        adjacency list). Note that it assumes that the two vertices are
        connected. It runs in $O(|V|+|E|)$ time.
        \code{graph/bfs.cpp}

        Another implementation that doesn't assume the two vertices are
        connected. If there is no path from the starting vertex to the ending
        vertex, a $-1$ is returned.
        \code{graph/bfs_visited.cpp}

    \subsection{Single-Source Shortest Paths}
        \subsubsection{Dijkstra's algorithm}
            An implementation of Dijkstra's algorithm. It runs in
            $\Theta(|E|\log{|V|})$ time.
            \code{graph/dijkstra.cpp}

        \subsubsection{Bellman-Ford algorithm}
            The Bellman-Ford algorithm solves the single-source shortest paths
            problem in $O(|V||E|)$ time. It is slower than Dijkstra's
            algorithm, but it works on graphs with negative edges and has the
            ability to detect negative cycles, neither of which Dijkstra's
            algorithm can do.
            \code{graph/bellman_ford.cpp}

    \subsection{All-Pairs Shortest Paths}
        \subsubsection{Floyd-Warshall algorithm}
            The Floyd-Warshall algorithm solves the all-pairs shortest paths
            problem in $O(|V|^3)$ time.
            \code{graph/floyd_warshall.cpp}

    \subsection{Strongly Connected Components}
        \subsubsection{Kosaraju's algorithm}
            Kosarajus's algorithm finds strongly connected components of a
            directed graph in $O(|V|+|E|)$ time.
            \code{graph/scc.cpp}

    \subsection{Minimum Spanning Tree}
        \subsubsection{Kruskal's algorithm}
            \code{graph/kruskals_mst.cpp}

    \subsection{Topological Sort}
        \subsubsection{Modified Depth-First Search}
            \code{graph/tsort.cpp}

    \subsection{Bipartite Matching}
        The alternating paths algorithm solves bipartite matching in $O(mn^2)$
        time, where $m$, $n$ are the number of vertices on the left and right
        side of the bipartite graph, respectively.
        \code{graph/bipartite_matching.cpp}

    \subsection{Maximum Flow}
        \subsubsection{Edmonds Karp's algorithm}
            An implementation of Edmonds Karp's algorithm that runs in
            $O(|V||E|^2)$. It computes the maximum flow of a flow network.
            \code{graph/edmonds_karps.cpp}

    \subsection{Minimum Cost Maximum Flow}
        An implementation of Edmonds Karp's algorithm, modified to find
        shortest path to augment each time (instead of just any path). It
        computes the maximum flow of a flow network, and when there are
        multiple maximum flows, finds the maximum flow with minimum cost.
        \code{graph/edmonds_karps_mcmf.cpp}

    \subsection{All Pairs Maximum Flow}
        \subsubsection{Gomory-Hu Tree}
        An implementation of the Gomory-Hu Tree. The spanning tree is constructed using Gusfield's algorithm
        in $O(|V| ^ 3)$ plus $|V|-1$ times the time it takes to calculate the maximum flow.
        If Edmonds Karp's algorithm is used to calculate the max flow the running time is $O(|V|^3 + |V|^2|E|^2)$.
        \code{graph/gomory_hu_tree.cpp}

\section{Strings}
    \subsection{The $Z$ algorithm}
        Given a string $S$, $Z_i(S)$ is the longest substring of $S$ starting
        at $i$ that is also a prefix of $S$. The $Z$ algorithm computes these
        $Z$ values in $O(n)$ time, where $n = |S|$. $Z$ values can, for
        example, be used to find all occurrences of a pattern $P$ in a string
        $T$ in linear time. This is accomplished by computing $Z$ values of $S
        = T P$, and looking for all $i$ such that $Z_i \geq |T|$.

        \code{strings/z_algorithm.cpp}

\section{Mathematics}
    \subsection{Fraction}
        A fraction (rational number) class. Note that numbers are stored in
        lowest common terms.
        \code{mathematics/fraction.cpp}

    \subsection{Binomial Coefficients}
        The binomial coefficient $\binom{n}{k} = \frac{n!}{k!(n-k)!}$ is the
        number of ways to choose $k$ items out of a total of $n$ items.
        \code{mathematics/nck.cpp}

    \subsection{Euclidean algorithm}
        The Euclidean algorithm computes the greatest common divisor of two
        integers $a$, $b$.
        \code{mathematics/gcd.cpp}

        The extended Euclidean algorithm computes the greatest common divisor
        $d$ of two integers $a$, $b$ and also finds two integers $x$, $y$ such
        that $a\times x + b\times y = d$.
        \code{mathematics/egcd.cpp}

    \subsection{Trial Division Primality Testing}
        An optimized trial division to check whether an integer is prime.
        \code{mathematics/is_prime.cpp}

    \subsection{Sieve of Eratosthenes}
        An optimized implementation of Eratosthenes' Sieve.
        \code{mathematics/prime_sieve.cpp}

    \subsection{Modular Multiplicative Inverse}
        A function to find a modular multiplicative inverse.
        \code{mathematics/mod_inv.cpp}

    \subsection{Modular Exponentiation}
        A function to perform fast modular exponentiation.
        \code{mathematics/mod_pow.cpp}

    \subsection{Chinese Remainder Theorem}
        An implementation of the Chinese Remainder Theorem.
        \code{mathematics/crt.cpp}

    \subsection{Formulas}
        \begin{itemize}
            \item Number of ways to choose $k$ objects from a total of $n$
                objects where order matters and each item can only be chosen
                once: $P^n_k = \frac{n!}{(n-k)!}$
            \item Number of ways to choose $k$ objects from a total of $n$
                objects where order matters and each item can be chosen
                multiple times: $n^k$
            \item Number of permutations of $n$ objects, where there are $n_1$
                objects of type $1$, $n_2$ objects of type $2$, \ldots, $n_k$
                objects of type $k$: $\binom{n}{n_1,n_2,\ldots,n_k} =
                \frac{n!}{n_1! \times n_2! \times \cdots \times n_k!}$
            \item Number of ways to choose $k$ objects from a total of $n$
                objects where order does not matter and each item can only be
                chosen once: \\ $\binom{n}{k} = \binom{n-1}{k-1} + \binom{n-1}k
                = \binom{n}{n-k} = \prod_{i=1}^k \frac{n-(k-i)}{i} =
                \frac{n!}{k!(n-k)!}, \binom n0 = 1, \binom 0k = 0$
            \item Number of ways to choose $k$ objects from a total of $n$
                objects where order does not matter and each item can be chosen
                multiple times: $f^n_k = \binom{n+k-1}{k} =
                \frac{(n+k-1)!}{k!(n-1)!}$
            \item Number of integer solutions to $x_1 + x_2 + \cdots + x_n = k$
                where $x_i \geq 0$: $f^n_k$
            \item Number of subsets of a set with $n$ elements: $2^n$
            \item $|A \cup B| = |A| + |B| - |A \cap B|$
            \item $|A \cup B \cup C| = |A| + |B| + |C| - |A \cap B| - |A \cap
                C| - |B \cap C| + |A \cap B \cap C|$
            \item Number of ways to walk from the lower-left corner to the
                upper-right corner of an $n\times m$ grid by walking only up
                and to the right: $\binom{n+m}{m}$
            \item Number of strings with $n$ sets of brackets such that the
                brackets are balanced: \\ $C_n = \sum_{k=0}^{n-1} C_kC_{n-1-k}
                = \frac{1}{n+1}\binom{2n}n$
            \item Number of triangulations of a convex polygon with $n$ points,
                number of rooted binary trees with $n+1$ vertices, number of
                paths across an $n\times n$ lattice which do not rise above the
                main diagonal: $C_n$
            \item Number of permutations of $n$ objects with exactly $k$
                ascending sequences or {\it runs}: \\ $\left\langle {n \atop k}
                \right\rangle = \left\langle {n \atop n - k - 1} \right\rangle
                = k\left\langle {n - 1 \atop k} \right\rangle + (n - k +
                1)\left\langle {n-1 \atop k-1} \right\rangle =
                \sum_{i=0}^{k}(-1)^i \binom{n+1}{i} (k+1-i)^n, \left\langle {n
                \atop 0} \right\rangle = \left\langle {n \atop n -1}
                \right\rangle = 1$
            \item Number of permutations of $n$ objects with exactly $k$
                cycles: $\left[n \atop k\right] = \left[n-1 \atop k-1\right] +
                (n-1)\left[n-1 \atop k\right]$
            \item Number of ways to partition $n$ objects into $k$ sets:
                $\left\{n \atop k\right\} = k\left\{n-1 \atop k\right\} +
                \left\{n-1 \atop k-1\right\}, \left\{n\atop 0\right\} =
                \left\{n\atop n\right\} = 1$
        \end{itemize}


\section{Geometry}
    \subsection{Primitives}
        Geometry primitives.
        \code{geometry/primitives.cpp}

    \subsection{Convex Hull}
        An algorithm that finds the Convex Hull of a set of points.
        \code{geometry/convex_hull.cpp}

    \subsection{Formulas}
        Let $a = (a_x, a_y)$ and $b = (b_x, b_y)$ be two-dimensional vectors.
        \begin{itemize}
            \item $a\cdot b = |a||b|\cos{\theta}$, where $\theta$ is the angle
                between $a$ and $b$.
            \item $a\times b = |a||b|\sin{\theta}$, where $\theta$ is the
                signed angle between $a$ and $b$.
            \item $a\times b$ is equal to the area of the parallelogram with
                two of its sides formed by $a$ and $b$. Half of that is the
                area of the triangle formed by $a$ and $b$.
        \end{itemize}


\section{Other Algorithms}
    \subsection{Binary Search}
        An implementation of binary search that finds a real valued root of the
        continous function $f$ on the interval $[a,b]$, with a maximum error of
        $\varepsilon$.
        \code{other/binary_search_continuous.cpp}

        Another implementation that takes a binary predicate $f$, and finds an
        integer value $x$ on the integer interval $[a,b]$ such that $f(x) \land
        \lnot f(x - 1)$.
        \code{other/binary_search_discrete.cpp}

    \subsection{2SAT}
        A fast 2SAT solver.
        \code{other/two_sat.cpp}

    \subsection{$n$th Permutation}
        A very fast algorithm for computing the $n$th permutation of the list
        $\{0,1,\ldots,k-1\}$.
        \code{other/nth_permutation.cpp}

    \subsection{Cycle-Finding}
        An implementation of Floyd's Cycle-Finding algorithm.
        \code{other/floyds_algorithm.cpp}

    \subsection{Dates}
        Functions to simplify date calculations.
        \code{other/dates.cpp}


\section{Useful Information}
    \subsection{Tips \&{} Tricks}
        \begin{itemize}
            \item How fast does our algorithm have to be? Can we use
                brute-force?
            \item Does order matter?
            \item Is it better to look at the problem in another way? Maybe
                backwards?
            \item Are there subproblems that are recomputed? Can we cache them?
            \item Do we need to remember everything we compute, or just the
                last few iterations of computation?
            \item Does it help to sort the data?
            \item Can we speed up lookup by using a map (tree or hash) or an
                array?
            \item Can we binary search the answer?
            \item Can we add vertices/edges to the graph to make the problem
                easier? Can we turn the graph into some other kind of a graph
                (perhaps a DAG, or a flow network)?
            \item Make sure integers are not overflowing.
            \item Is it better to compute the answer modulo $n$? Perhaps we can
                compute the answer modulo $m_1,m_2,\ldots,m_k$, where
                $m_1,m_2,\ldots,m_k$ are pairwise coprime integers, and find
                the real answer using CRT?
            \item Are there any edge cases? When $n=0, n=-1, n=1, n=2^{31}-1$
                or $n=-2^{31}$? When the list is empty, or contains a single
                element? When the graph is empty, or contains a single vertex?
                When the graph contains self-loops?  When the polygon is
                concave or non-simple?
            \item Can we use exponentiation by squaring?
        \end{itemize}

    \subsection{Fast Input Reading}
        If input or output is huge, sometimes it is beneficial to optimize the
        input reading/output writing. This can be achieved by reading all input
        in at once (using fread), and then parsing it manually. Output can also
        be stored in an output buffer and then dumped once in the end (using
        fwrite). A simpler, but still effective, way to achieve speed is to use
        the following input reading method.
        \code{tricks/fast_input.cpp}

    \subsection{Worst Time Complexity}
    \begin{center}
        \begin{tabular}{c|c|c}
            $n$ & Worst AC Algorithm & Comment \\
            \hline
            $\leq 10$ & $O(n!), O(n^6)$ & e.g. Enumerating a permutation \\
            $\leq 15$ & $O(2^n\times n^2)$ & e.g. DP TSP \\
            $\leq 20$ & $O(2^n), O(n^5)$ & e.g. DP + bitmask technique \\
            $\leq 50$ & $O(n^4)$ & e.g. DP with 3 dimensions + $O(n)$ loop, choosing  $_nC_k=4$ \\
            $\leq 10^2$ & $O(n^3)$ & e.g. Floyd Warshall's \\
            $\leq 10^3$ & $O(n^2)$ & e.g. Bubble/Selection/Insertion sort \\
            $\leq 10^5$ & $O(n\log_2{n})$ & e.g. Merge sort, building a Segment tree \\
            $\leq 10^6$ & $O(n), O(\log_2{n}), O(1)$ & Usually, contest problems have $n\leq10^6$ (e.g. to read input) \\
        \end{tabular}
    \end{center}

    \subsection{Bit Hacks}
        \begin{itemize}
            \item \texttt{n \&{} -n} returns the first set bit in $n$.
            \item \texttt{n \&{} (n - 1)} is $0$ only if $n$ is a power of two.
            \item \texttt{snoob(x)} returns the next integer that has the
                same amount of bits set as \texttt{x}. Useful for iterating
                through subsets of some specified size.
                \code{tricks/snoob.cpp}
        \end{itemize}

\end{document}
